\documentclass{acm_proc_article-sp}
\usepackage[utf8]{inputenc}

\begin{document}
\title{E-passport: Security attacks}
\subtitle{[State of art]}

\numberofauthors{2} 
\author{
\alignauthor
Aquiles Ricardo TORRES ALVAREZ\titlenote{Electrical student at Universidad del Norte - Colombia. Double major student of Telecommunication at Ensimag - France}\\
       \affaddr{Grenoble INP - ENSIMAG}\\
       \affaddr{Grenoble, Francia}\\
       \email{torresaa@}
% 2nd. author
\alignauthor
Jorge Luis GULFO MONSALVE\titlenote{Electrical student at Universidad del Norte - Colombia. Double major student of Telecommunication at Ensimag - France}\\
       \affaddr{Grenoble INP - ENSIMAG}\\
       \affaddr{Grenoble Francia}\\
       \email{gulfomoj@}
}

\maketitle
\begin{abstract}

\end{abstract}

\terms{Theory}

\keywords{passport, MRT, RFID, ICAO, eavesdropping, BAC, attack, cryptography}

\section{Introduction}

\section{Technical Background}

\subsection{RFID}
The way the E-passport works is by communicating with a RF reader which emits a wave that feeds the IC contactless chip (as mentioned at \cite{NM12}, because is the only one that meets the ICAO’s requirements) located at the passport which is a Radio Frequency ID (because it is meant to be unique) or tag. Once he is feeded the exchange of information begins but without protection any other RF reader around the passport could scan that information so knowing all those security issues and privacy threats attached to the data exchanged between the e-passport and the reader, the ICAO (International Civil Aviation Organization) , which is the organization that sets all the protocols and standards for the e-passport, has established a set of protocols such as Passive Authentication, Basic Access Control, Active Authentication and Extended Access Control for encrypting the information exchanged between the RFID chip and the reader and keeping the privacy and security of the e-passport holder.

As mentioned at \cite{NM12} the first protocol to treat was PA, which signs the passport with a public key of the issuing country in order to prove the integrity and authenticity of the data.

\subsection{BAC Protocol}
As mentioned at \cite{CLPS07}, BAC which is a protocol that prevents skimming by encrypting the data with two symmetric keys (K-ENC and K-MAC) that are derived from the holder’s birthdate, the passport’s expiry date and the alphanumeric passport number. Once the reader is authenticated by reading those keys an unique session key is generated for accessing the MRZ (Machine Readable Zone) in which all the passport and biometrical information of the holder are contemplated. 

COPIAR IMAGEN DE DOCUMENTO E-PASSPORT GLOBAL TRACEABILITY PAG 6. 

So as we can see at the figure once the reader receives the challenge from the tag, he answers by encrypting E and M with the keys K-ENC and K-MAC so that the reader be able to prove the knowledge of the keys from the reader, when the tag proves the authenticity a session key KS-Seed is generated for later transmit the MRZ data.

\subsection{Ative Authentication}
As mentioned at \cite{JUAR2005}, in order to prevent cloning the AA protocol was created. This protocol is more an anti-cloning feature because here the passport must prove to the reader that he has a private key as a response to a challenge previously received.

\subsection{Extended Access Control}
In \cite{NM12} is mentioned that in next generation e-passports EAC has being used instead of BAC because before proceeding with the same steps as BAC he verifies the authenticity of the reader by a certificate validation from the issuing passport’s country, certificate which must be transmitted in a safe way.

%\subsection{Citations}
%Citations to articles \cite{bowman:reasoning, clark:pct, braams:babel, herlihy:methodology},
%conference

\section{E-passports Attacks}

\subsection{Cryptographic Weaknesses}
As we know, the \textit{Basic Access Control (BAC)} protocol establishes a secured channel between 
the \textit{RFID} tag and the reader in order to provided confidentiality and integrity to the 
data communication. \textit{BAC protocol} generates the encryption and authentication keys from the data 
present in passport's \textit {MRZ} (number, expiration, date of birth ) and it is demonstrable 
that the entropy of this data don’t reach at least the 80 bits suggested and can be worse 
with simple observations \cite{JUAR2005} \cite{02COPA}.\\
Liu, Kasper and others \cite{02COPA} had made a complexity analysis of the key space focused on 
demonstrate the low entropy of \textit{BAC keys} for two main passport’s nationality: Germany, 
Netherlands but easily extensible to other countries. The low entropy of key is caused 
first by the use of mainly numeric characters on passport number, 
second because of the stochastic dependency between passport number and its expiration 
date, and finally due to dependency of publicly available personal data. The analysis 
showed that in the best case, we know only public information and issuing state, the 
entropy reach \textbf {52.8 bits} for the German’s passports but it can fall to \textbf {20.4 bits} for 
Netherlands passports if we have a \textit{BAC keys database}. This low entropy is more disturbing 
when at the of the study they estimate the time to find the \textit{MRZ} in 25h and less than 185ms 
respectively.


\subsection{Traceability}

\subsection{Physical-layer Weaknesses}

\subsection{Cloning}
As mentioned in 333333, once the reader and the tag have passed BAC, AA is implemented. As it was already mentioned the objective of this protocol is to the prevent cloning, so once the reader sends the challenge to the tag it responds immediately responds with a WAIT message leaving a gap of five seconds which finally are about four seconds because the tag takes approximately one second to perform the private key validation. So a possible scenario (as dictated at 33333) is when somebody (TERMINAR DE CONTAR LA HISTORIA).
\section{Conclusions}

%\end{document}  % This is where a 'short' article might terminate

HELLLO DFDFDFDFDF~\cite{Chothia10}

\bibliographystyle{abbrv}
\bibliography{sigproc}  % sigproc.bib is the name of the Bibliography in this case
% You must have a proper ".bib" file
%  and remember to run:
% latex bibtex latex latex
% to resolve all references
%
% ACM needs 'a single self-contained file'!
%

\balancecolumns
% That's all folks!
\end{document}
