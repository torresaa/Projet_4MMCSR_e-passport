\documentclass{acm_proc_article-sp}
\usepackage[utf8]{inputenc}

\begin{document}
\title{E-passport: Security attacks}
\subtitle{[State of art]}

\numberofauthors{2} 
\author{
\alignauthor
Aquiles Ricardo TORRES ALVAREZ\titlenote{Electrical student at Universidad del Norte - Colombia. Double major student of Telecommunication at Ensimag - France}\\
       \affaddr{Grenoble INP - ENSIMAG}\\
       \affaddr{Grenoble, Francia}\\
       \email{torresaa@}
% 2nd. author
\alignauthor
Jorge Luis GULFO MONSALVE\titlenote{Electrical student at Universidad del Norte - Colombia. Double major student of Telecommunication at Ensimag - France}\\
       \affaddr{Grenoble INP - ENSIMAG}\\
       \affaddr{Grenoble Francia}\\
       \email{gulfomoj@}
}

\maketitle
\begin{abstract}

\end{abstract}

\terms{Theory}

\keywords{passport, MRT, RFID, ICAO, eavesdropping, BAC, attack, cryptography}

\section{Introduction}

\section{Technical Background}

\subsection{RFID}

\subsection{BAC Protocol}

\subsection{Ative Authentication}

\subsection{Extended Access Control}

%\subsection{Citations}
%Citations to articles \cite{bowman:reasoning, clark:pct, braams:babel, herlihy:methodology},
%conference

\section{E-passports Attacks}

\subsection{Cryptographic Weaknesses}
As we know, the \textit{Basic Access Control (BAC)} protocol establishes a secured channel between 
the \textit{RFID} tag and the reader in order to provided confidentiality and integrity to the 
data communication. \textit{BAC protocol} generates the encryption and authentication keys from the data 
present in passport's \textit {MRZ} (number, expiration, date of birth ) and it is demonstrable 
that the entropy of this data don’t reach at least the 80 bits suggested and can be worse 
with simple observations \cite{JUAR2005} \cite{02COPA}.\\
Liu, Kasper and others \cite{02COPA} had made a complexity analysis of the key space focused on 
demonstrate the low entropy of \textit{BAC keys} for two main passport’s nationality: Germany, 
Netherlands but easily extensible to other countries. The low entropy of key is caused 
first by the use of mainly numeric characters on passport number, 
second because of the stochastic dependency between passport number and its expiration 
date, and finally due to dependency of publicly available personal data. The analysis 
showed that in the best case, we know only public information and issuing state, the 
entropy reach \textbf {52.8 bits} for the German’s passports but it can fall to \textbf {20.4 bits} for 
Netherlands passports if we have a \textit{BAC keys database}. This low entropy is more disturbing 
when at the of the study they estimate the time to find the \textit{MRZ} in 25h and less than 185ms 
respectively.


\subsection{Traceability}

\subsection{Physical-layer Weaknesses}

\subsection{Cloning}

\section{Conclusions}

%\end{document}  % This is where a 'short' article might terminate

HELLLO DFDFDFDFDF~\cite{Chothia10}

\bibliographystyle{abbrv}
\bibliography{sigproc}  % sigproc.bib is the name of the Bibliography in this case
% You must have a proper ".bib" file
%  and remember to run:
% latex bibtex latex latex
% to resolve all references
%
% ACM needs 'a single self-contained file'!
%

\balancecolumns
% That's all folks!
\end{document}
